%%%%%%%%%%%%%%%%%%%%%%%%%%%%%%%%%%%%%%%%%
% Short Sectioned Assignment
% LaTeX Template
% Version 1.0 (5/5/12)
%
% This template has been downloaded from:
% http://www.LaTeXTemplates.com
%
% Original author:
% Frits Wenneker (http://www.howtotex.com)
%
% License:
% CC BY-NC-SA 3.0 (http://creativecommons.org/licenses/by-nc-sa/3.0/)
%
%%%%%%%%%%%%%%%%%%%%%%%%%%%%%%%%%%%%%%%%%



%----------------------------------------------------------------------------------------
%	PACKAGES AND OTHER DOCUMENT CONFIGURATIONS
%----------------------------------------------------------------------------------------
\documentclass[paper=a4, fontsize=11pt]{scrartcl} % A4 paper and 11pt font size
\usepackage[utf8]
{inputenc}
\usepackage[T1]{fontenc} % Use 8-bit encoding that has 256 glyphs
%\usepackage{fourier} % Use the Adobe Utopia font for the document - comment this line to return to the LaTeX default
\usepackage[english]{babel} % English language/hyphenation
\usepackage{amsmath,amsfonts,amsthm} % Math packages
\usepackage[hidelinks]{hyperref}
\usepackage{natbib} % bibliography

\usepackage{lipsum} % Used for inserting dummy 'Lorem ipsum' text into the template

\usepackage{sectsty} % Allows customizing section commands

\usepackage{titlesec}

\usepackage{pgfplots}
\usepgfplotslibrary{dateplot}

\allsectionsfont{\centering \normalfont\scshape} % Make all sections centered, the default font and small caps

\usepackage{fancyhdr} % Custom headers and footers
\pagestyle{fancyplain} % Makes all pages in the document conform to the custom headers and footers
\fancyhead{} % No page header - if you want one, create it in the same way as the footers below
\fancyfoot[L]{} % Empty left footer
\fancyfoot[C]{} % Empty center footer
\fancyfoot[R]{\thepage} % Page numbering for right footer
\renewcommand{\headrulewidth}{0pt} % Remove header underlines
\renewcommand{\footrulewidth}{0pt} % Remove footer underlines
\setlength{\headheight}{13.6pt} % Customize the height of the header

\numberwithin{equation}{section} % Number equations within sections (i.e. 1.1, 1.2, 2.1, 2.2 instead of 1, 2, 3, 4)
\numberwithin{figure}{section} % Number figures within sections (i.e. 1.1, 1.2, 2.1, 2.2 instead of 1, 2, 3, 4)
\numberwithin{table}{section} % Number tables within sections (i.e. 1.1, 1.2, 2.1, 2.2 instead of 1, 2, 3, 4)

\setlength\parindent{0pt} % Removes all indentation from paragraphs - comment this line for an assignment with lots of text

%----------------------------------------------------------------------------------------
%	TITLE SECTION
%----------------------------------------------------------------------------------------

\newcommand{\horrule}[1]{\rule{\linewidth}{#1}} % Create horizontal rule command with 1 argument of height

\title{	
\normalfont \normalsize 
\textsc{Australian National University} \\ [25pt] % Your university, school and/or department name(s)
\horrule{0.5pt} \\[0.4cm] % Thin top horizontal rule
\huge A MC-AIXI-CTW Implementation\\ Group Project \\ % The assignment title
\horrule{2pt} \\[0.5cm] % Thick bottom horizontal rule
}

\author{Johannes Kirschner\\ Kerry Olesen\\ Jesse Wu} % Your name

\date{\normalsize\today} % Today's date or a custom date


%\titleformat{\section}{\large\centering\normalfont\scshape}{}{0em}{}
%\titleformat{\subsection}{\centering\normalfont}{\roman{subsection})}{0em}{}

  

\begin{document}

\maketitle % Print the title

\section{Introduction}
The AIXI model \cite{Hutter:04uaibook} is an attempt to solve the general AI problem. The AIXI agent interacts with an environment in cycles. Denote by $\mathcal{A}, \mathcal{O}$ and $\mathcal{R}$ an action, observation and reward space respectively. In each cycle, AIXI takes an action $a \in \mathcal{A}$ and receives an observation $o \in \mathcal{O}$ and a reward $r \in \mathcal{R}$. The goal of the agent is to maximize the total future reward. AIXI does not require any previous knowledge of an environment, actions are chosen based on past perceptions, which are used to build a model of the environment. More specifically, AIXI chooses in cycle $k$ an action
\[ a_k = \arg \max_{a_k} \sum_{o_kr_k} \dots \max_{a_m} \sum_{o_m r_m} (r_k + \dots + r_m)\xi(o_1r_1\dots o_m r_m|a_1\dots a_m) \]
TODO: explain what $\xi$ is, and why it is good (because it is so large)

Unfortunately the AIXI model is incomputable. For all practical applications, the agent must be approximated. One approach in approximating AIXI is the MC-AIXI-CTW \cite{VNHS09} model. Here the expectimax search is solved by an Monte-Carlo approach. The UTC \cite{UCT} algorithm is used to balance exploration and exploitation. The class of environment models used in the implementation is a mixture of $d$-th order Markov Decision Process. Notably, Context Tree Weighting allows efficient linear time computation of this rather general class of models \cite{CTW}.
In the following we present our implementation of the MC-AIXI-CTW model. In Section~\ref{usr} we explain how to use the program and specify different options. Section~\ref{results} describes the results of the model on several experimental environments.


\section{\label{usr}User Manual}

Our approximation of aixi is written in C++ and requires g++ for compilation.

\subsection{Setup}

\setlength\parindent{20pt}
\noindent Compile:\\
\indent \textit{cd aixi}\\
\indent \textit{make}

\bigskip

\noindent Run:\\
\indent \textit{./aixi file.conf [-{}-option1=value1 -{}-option2=value2 \dots]}
\setlength\parindent{0pt}

Include trained ctw data?? I think this is a good idea Johannes

\subsection{Configuration Options}

Options can be either specified in the configuration file or passed directly as --option=value to the program. Several configuration files are available, each specifies a particular environment and a set of default options. 

\subsubsection*{Available Options}

\paragraph{-{}-environment=env} Specifies the environment. Available environments are
\begin{itemize}
\itemsep0pt
\renewcommand\labelitemi{--}
    \item biased\textunderscore rock\textunderscore paper\textunderscore scissor
    \item coinflip
    \item kuhn\textunderscore poker
    \item pacman
    \item tiger
\end{itemize}

\paragraph{-{}-mc-timelimit=N} The number $N$ of MC simulations per cycle.
\paragraph{-{}-write-ct=file} Write CTW to file before agent termination.
\paragraph{-{}-load-ct=file} Specifies a (trained) CTW for the agent to load at initialisation.
\paragraph{-{}-log=file}
\paragraph{-{}-terminate-age=N} The number $N$ of agent/environment interaction cycles.
\paragraph{-{}-exploration=P} Probability $0 \leq P \leq 1$ that a action is chosen randomly.
\paragraph{-{}-explore-decay=D} Decay $0 \leq D \leq 1$ of exploration constant. $P$ is multiplied by $D$ in each cycle.
%\paragraph{-{}-}

\section{Code Documentation}
Do we need this??

\section{\label{results}Experimental Results}
\subsection{Experimental Setup}

\begin{itemize}
    \item List/Make table with configurations used for each environment.
    \item Present results of each environment
    \item List hardware (cpu/clock speed/cache/ram)
\end{itemize}

\subsection{Results}
Present Results/Graphs of each environment. Maybe mention optimal result and/or scale results accordingly

\subsection{Further Experiments}
How does the exploration constant affect learning? (tiger env)\\
How does aixi handle a change of the environment?\\
-- Compare 0.3 coinflip to 0.7 coinflip\\
-- biased rps vs tiger\\

%sample graph

\begin{figure}
\input{plots/coinflip.tex}
\caption{\label{tab:setup}coinflip environment}
\end{figure}
\end{document}

\begin{figure}[tb]
\centering
\begin{tabular}{l|c|c|c|c|c}
Domain & CTW depth & $m$ & $\epsilon$ & $\gamma$ & $\rho$UCT Simulations\\\hline
%TODO: fill in
Biased Rock-Paper-Scissor & \\
Coinflip & \\
Kuhn-Poker & \\
Pacman & \\
Tiger & \\
\end{tabular}
\caption{\label{tab:setup}Agent configurations}
\end{figure}

\subsection{Discussion}
How well did it do.

Include results to do with forgetting past model - changing environments

Include statistics about cycles required for optimal performance, time per cycle as in the VNHS paper \cite{VNHS09}.

TODO: Does anyone have a bibtex library? Or shall we do references manually? We probably won't have many.


\bibliographystyle{alpha}
\bibliography{references}

\end{document}
